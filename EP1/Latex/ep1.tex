\documentclass{article}
\usepackage{tikz}
\usepackage[utf8]{inputenc}

\title{MAC0350 - EP1}
\date{}
\author{
	André Ferrari Moukarzel \\ 9298166
	\and 
	Gabriel Sarti Massukado \\ 10284177
	\and
	Matheus Lima Cunha \\ 10297755
}

\begin{document}
  \pagenumbering{gobble}
  \maketitle
  \newpage
  \pagenumbering{arabic}
  
  \section{Item 2}
    \subsection{Generalização: Pessoa}
        \quad A entidade "Pessoa" generaliza de forma sobreponível as entidades regulares "Professor", "Aluno" e "Administradores". \\
        \quad A pessoa possui CPF e nome como atributos, sendo que CPF é chave primária.
        \subsubsection{Restrições}
            \begin{itemize}
                \item CPF: Número de Cadastro de Pessoa Física da pessoa. Deve ser uma sequência de 11 digitos constituindo uma sequência válida seguinte as regras da Receita Federal. Não pode ser NULL.
                \item Nome: Nome completo do individuo conforme escrito em sua certidão de nascimento. É composto de letras, podendo conter no máximo 200 caracteres. 
            \end{itemize}
            
  	\subsection{Entidade Regular: Professor}
  		\quad O professor é uma entidade especializada de pessoa onde armazenaremos todos os dados das instâncias de cada um dos professores da USP presentes na base de dados. \\
  		\quad Um professor possui os atributos NUSP e Departamento, sendo que NUSP é chave primária.  
  		\subsubsection{Restrições}
  		    \begin{itemize}
  		        \item NUSP: Número de identificação da USP do professor. É composta de até 9 dígitos e, por ser a chave primária, não pode ser nulo. 
  		        \item Dpt: Identificação de qual o departamento o professor pertence. Os valores que pode assumir são: MAT, MAC, MAE ou MAP.
  		    \end{itemize}
  		    
  	\subsection{Entidade Regular: Aluno}
  	    \quad O aluno é uma entidade especializada de pessoa onde armazenaremos todos os dados das instâncias de cada um dos alunos da USP presentes na base de dados. \\
  	    \quad Uma instância de aluno possui o atributo NUSP, que é chave primária.
  	    \subsubsection{Restrições}
  	        \begin{itemize}
  		        \item NUSP: Número de identificação da USP do professor. É composta de até 9 dígitos e, por ser a chave primária, não pode ser nulo. 
  		    \end{itemize}
  		    
  	\subsection{Entidade Regular: Administrador}
  	    \quad O administrador é uma entidade especializada de pessoa em que armazenaremos todos os dados  das instancias de cada um dos administradores na base de dados. \\
  	    \quad Uma instância de administrador possui os atributos...?
  	    \subsubsection{Restrições}
  	        \begin{itemize}
  	            \item 
  	        \end{itemize}
  	        
  	\subsection{Entidade Regular: Disciplina}
  	    \quad A disciplina é uma entidade em que armazenaremos todos os dados das instâncias de cada uma das disciplinas na base de dados. \\
  	    \quad Uma disciplina possui um nome, um código, um número de créditos trabalho, um número de créditos aula e período ideal, sendo que o código da matéria é sua chave primária.
  	    \subsubsection{Restrições}
  	        \begin{itemize}
  		        \item Nome\_Disc: Nome por extenso da disciplina. É uma sequência de até 80 caracteres.
  		        \item Cod\_Disc: O código da matéria, contendo 3 letras seguidas de 4 digitos numéricos. Não pode ser NULL.
  		        \item Cred\_Trab: Número de créditos trabalho concedidos pela disciplina. Deve ser um número maior ou igual a zero.
  		        \item Cred\_Aula: Número de créditos aula concedidos pela disciplina. Deve ser um número maior ou igual a zero.
  		        \item Per\_Ideal: O período ideal para essa disciplina ser cursada. Deve ser um número maior que zero e menor que 11, dado que nenhum curso da USP tem duração de mais de 10 semestres.
  		    \end{itemize}
  		    
  	\subsection{Entidade Regular: Curso}
  	
  	
  	
  	
  	
  	\subsection{Entidade Regular: Trilha}
  	    \quad A trilha é uma entidade regular que representa um conjunto de módulos que representam a especialização em uma área do conhecimento da computação. Possui um nome.
  	    \subsubsection{Restrições}
  	        \begin{itemize}
  	            \item Nome: Nome por extenso da trilha. É uma sequência de até 80 caracteres.
  		    \end{itemize}
  		    
  	\subsection{Entidade Regular: Módulo}
  	    \quad Um módulo é um conjunto de materia especificas que representam uma pequena área do conhecimento de uma trilha. Cada módulo possui um nome e o número de materias a serem cumpridas.
  	    \begin{itemize}
  	            \item Nome: Nome por extenso da trilha. É uma sequência de até 80 caracteres.
  	            \item Num\_Mat: Número de minimo materiais que devem ser concluidas para completar o módulo. Deve ser um número maior que zero composto de até 2 digitos. 
  		    \end{itemize}
  		    
  	\subsection{Entidade Regular: Usuário}
  	    \quad O usuário é uma entidade regular em que armazenamos cada uma das instâncias dos usuários do serviço fornecido pelo nosso banco de dados. \\
  	    \quad Uma instância de usuário tem como atributos...?
  	    \subsubsection{Restrições}
  	        \begin{itemize}
  	            \item 
  	        \end{itemize}
  	        
  	\subsection{Entidade Regular: Perfil}
  	
  	
  	
  	\subsection{Entidade Regular: Serviço}
  	
  	\subsection{Entidade Regular: Obrigatoria}
  	
  	\subsection{Entidade Regular: Optativa}
  	
  	\subsection{Entidade Regular: Livre}
  	
  	\subsection{Entidade Regular: Eletiva}
  	
  	
  	\subsection{Agregação: Grade}
  	    \quad Uma grade é uma agregação das entidades "Disciplina" e "Curso" e da relação "rel\_dis\_curso" em que armazenamos o conjunto de disciplinas pertencentes a um mesmo curso que representam a grade deste curso. 
  	    \subsubsection{Restrições}
  	        \begin{itemize}
  		        \item 
  		    \end{itemize}
  		    
  	\subsection{Agregação: Oferecimento}
  	    \quad Um oferecimento é uma agregação das entidades "Professor" e "Disciplina" e da relação "Ministra" em que armazenamos as instâncias de oferecimentos de certas disciplinas em um semestre. \\
  	    \quad Uma instância de oferecimento tem como atributos turma, semestre, ano (que juntos atuam como chave primária) e horário. 
  	    \subsubsection{Restrições}
  	        \begin{itemize}
  	            \item turma: inteiro que indica um grupo de alunos que está cursando o oferecimento. Não pode ser NULL.
  	            \item semestre: inteiro que indica o semestre em que ocorre o oferecimento. Deve ser "1" ou "2".
  	            \item ano: inteiro que indica o ano em que ocorre o oferecimento. Deve ser um inteiro maior que 2000.
  	            \item horario: lista de tuplas que indica em que horarios as aulas do oferecimento ocorrem. Cada tupla contém um \textit{datatime} que representa o horário de início da aula, um \textit{datatime} que representa o horário de término da aula e um inteiro no intervalo [1, 7] que representa o dia da semana. Nenhum desses valores pode ser NULL e o horário deve possuir ao menos uma tupla.
  	         \end{itemize}
  	    
  	\subsection{Relação: pe\_us}
  	    \quad Relação que representa a posse de uma conta por uma pessoa. Relação 1 para 1 entre as entidades "Pessoa" e "Usuário".
  	    
  	\subsection{Relação: us\_pf}
  	
  	\subsection{Relação: pf\_se}
  	
  	\subsection{Relação: ministra}
  	    \quad Relação que representa a ministração de uma disciplina por um professor. Um professor pode ministrar múltiplas disciplinas, enquanto uma disciplina só pode ser ministrada por apenas um professor, configurando uma relação 1:N.
  	\subsection{Relação: cursa}
  	    \quad Relação que representa um aluno matriculado em um oferecimento de uma disciplina. Um aluno pode cursar múltiplos oferecimentos e um mesmo oferecimento pode ser cursado por múltiplos alunos, configurando uma relação N:M. \\
  	    \quad Esta relação possui os atributos nota e presença.
  	    \subsubsection{Restrições}
  	        \begin{itemize}
  	            \item Nota: Nota de um aluno em um oferecimento. A nota é um valor de ponto flutuante maior ou igual a 0.0 e menor ou igual a 10.0.
  	            \item Presença: Presença total de um aluno em um oferecimento que cursa. É um valor de ponto flutuante maior ou igual a 0.0 e menor ou igual a 1.0.
  	         \end{itemize}
  	\subsection{Relação: planeja}
  	    \quad Relação que representa um aluno com pedido de participação em uma disciplina. Um aluno pode planejar cursar múltiplas disciplinas e uma mesma disciplina pode ser alvo de planejamento de múltiplos alunos, configurando uma relação N:M.
  	\subsection{Relação: administra}
  	    \quad jajaja me gustan las papas
  	    \quad si por supuesto chaval
  	    
  	\subsection{Relação: rel\_dis\_cur}
  	
  	\subsection{Relação: op\_mod}
  	    \quad Relação que representa a participação de uma disciplina optativa em um módulo. Uma disciplina pode estar em N módulos e um módulo pode conter M disciplinas.
  	
  	\subsection{Relação: tr\_mod}
  	    \quad Relação que representa a inclusão de um módulo em uma trilha. Um módulo só pode ser parte de uma trilha, entretanto uma trilha pode conter N módulos.

\end{document}